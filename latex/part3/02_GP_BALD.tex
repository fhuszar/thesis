%!TEX root = ../main.tex
foo

\section{Gaussian process classification}

\subsection{Approximate BALD for GPC}
\subsection{Results}
\subsection{Conclusions}

\section{Preference elicitation}

In this section I address the practical problem of learing users' preferences.

Many approaches -- such as traditional market research surveys, restaurant review websites, DVD rental websites, etc -- require human respondents to give ratings of items an absolute scale. In surveys .
There are multiple problems with this approach.
\begin{enumerate}
	\item People's baseline level on the absolute scale may differ. One person's 4 star rating may describe the same level of satisfaction as someone else's 5 star rating.
	\item The variance of responses may also differ across people: some more conservative reviewers would never write an the extreme 1 star or 5 star reviews, while others' opinions may be more polarised.
	\item To give informed ratings, the user has to know the distribution of the quality of items ahead of time. They may prefer not to give a maximal, 5-star rating to an item, because they don't know if better items exist. Others may give 5-star to a mediocre item, because they have never seen to a better one.
\end{enumerate} 

The problem has widespred applications in a variety of domains.
\begin{itemize}
	\item on e-commerce website, learning about users preferences of item-price pairs, or learning about brand preferences. Would a user prefer a cheap item with no discount to a more expensive item with a slight discount? These preferences can then be exploited to maximise user satisfaction and to drive profit
	\item in social media, peerindex.com used 
	\item Learning about attractiveness, and the features that determine attractiveness. 
	\item Calibration of sound quality of hearing aid. 
	\item Computer graphics, finding parameter combination of lighting and 
\end{itemize}

\subsection{Relation to Gaussian process classification}

\subsection{Collaborative preference learning}
